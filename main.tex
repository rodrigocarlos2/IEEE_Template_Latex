
\documentclass[conference]{IEEEtran}
\IEEEoverridecommandlockouts
% The preceding line is only needed to identify funding in the first footnote. If that is unneeded, please comment it out.
\usepackage{cite}
\usepackage{amsmath,amssymb,amsfonts}
\usepackage{algorithmic}
\usepackage{graphicx}
\usepackage{textcomp}
\usepackage[portuguese,english]{babel}
\renewcommand\IEEEkeywordsname{Keywords}
\usepackage[utf8]{inputenc}

\def\BibTeX{{\rm B\kern-.05em{\sc i\kern-.025em b}\kern-.08em
    T\kern-.1667em\lower.7ex\hbox{E}\kern-.125emX}}
\begin{document}

\title{ Análise Crítica do Artigo Comparativo entre Ferramentas para Construção de
um Ambiente Virtual 3D\\}

\author{\IEEEauthorblockN{Davi Luis de OLiveira}
\IEEEauthorblockA{\textit{Universidade Federal do Piauí(UFPI)} \\
Picos,Piauí \\
daviluis323@gmail.com
}
\and
\IEEEauthorblockN{André Lucas da Costa Soares}
\IEEEauthorblockA{\textit{Universidade Federal do Piauí(UFPI)} \\
Picos,Piauí \\
andrelukas91@hotmail.com
}

\and
\IEEEauthorblockN{Wildyson Dantas dos Santos}
\IEEEauthorblockA{\textit{Universidade Federal do Piauí(UFPI)} \\
Picos,Piauí \\
wildyson.santos@gmail.com
}
}

\maketitle
\section*{Estrutura do artigo}
O artigo proposto para análise crítica possui a seguinte estrutura:
\begin{itemize}
\item INTRODUÇÃO
\item GAME ENGINES
\item AS ITACOATIARAS DO INGÁ
\item COMPARATIVO
\item CONCLUSÃO
\end{itemize}
O primeiro erro encontrado na estrutura do artigo é que existe uma seção para games engines mas não existe uma seção específica para o X3D, sendo umas das ferramentas comparadas no artigo. Não seria necessário uma seção específica para o lugar itacoatiaras do ingá, já que foco é comparativo de ferramentas de construção 3D ou seja uma breve explicação sobre o lugar já seria o suficente.
\section*{Decisões e Justificativas do artigo}
O comparativo possuem como objetivo apresentar as vantagens e desvantagens de usar games engines para a construção de um ambiente virtual 3D, usando para isso características qualitativas. O problema é que ele compara duas ferramentas que possuem as algumas funções iguais mais possuem formas de fazer ambientes 3D de um jeito diferente.
\section*{Problemas na Avaliação}
É comparaDO o unity que é uma game engine ou seja é um software que inclui motor gráfico para renderizar gráficos em 2D ou 3D, motor de física para detectar colisões e fazer animações, além de suporte para sons, inteligência artificial, gerenciamento de arquivos, programação, entre outros.com o X3D que é um padrão aberto baseado em XML que permite a criação de ambientes virtuais. É possível observar que os objetivos desse artigo são tendenciosos, porque é sempre citado as vantagens do unity e dando pouco foco no X3D. Também não possui uma abordagem sistemática, sendo não existe escolha de cargas de trabalho adequado para a comparação e os dados que são apresentados não são analisados corretamente dando sempre vantagem ao unity. As métricas usadas são qualitativas e não quantitativas, devido a isso não quantificar o desempenho dos sistemas abordados no comparativio.\\

Os autores do artigo cometeram um dos grandes erros de um estudo comparativo, pois compararam ferramentas de natureza diferente. Apesar de terem funções parecidas, a forma de utilização é totalmente diferente, o X3D é um padrão aberto para distribuição de conteúdo 3D e o Unity 3D é uma game engine. Na análise ainda foi usada uma versão paga do Unity 3D, o que torna a comparação totalmente injusta.\\
	 	 	
Na análise, os autores esconderam as limitações que o software possui. No momento que acontece a omissão desses fatos, fazem com que a análise seja corrompida. Durante o comparativo, eles colocaram foco apenas no Unity 3d, apresentando recursos que a versão paga possui, sendo bastante tendencioso nesse aspecto. Outro problema encontrado, foi a falta da análise de sensibilidade, onde o autor colocou ênfase apenas nos resultados. Vale ressaltar também, que o artigo deveria apresentar o resultados na forma de fatos e não como evidências.\\

\selectlanguage{portuguese}
\begin{thebibliography}{00}
\bibitem{b1}BATISTA, Thiago; MACHADO, Liliane; COSTA, Thaíse. Comparativo entre Ferramentas para Construção de um Ambiente Virtual 3D.
\end{thebibliography}


\end{document}
